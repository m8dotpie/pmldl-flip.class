\documentclass[
    11pt, % Set the default font size, options include: 8pt, 9pt, 10pt, 11pt, 12pt, 14pt, 17pt, 20pt
    %
    aspectratio=169, ]{beamer}% Uncomment to set the aspect ratio to a 16:9 ratio which matches the aspect ratio of 1080p and 4K screens and projectors

%\graphicspath{{Images/}{./}} % Specifies where to look for included images (trailing slash required)
\usepackage{booktabs} % Allows the use of \toprule, \midrule and \bottomrule for better rules in tables

%\usepackage{appendixnumberbeamer} %If you want a separate slide counter for your appendix

%%% To add citations
\usepackage[style=authoryear, backend=bibtex]{biblatex}
\usepackage{svg}
\usepackage{derivative}
\usepackage{multicol}
\addbibresource{auto.bib}

%%% Customize Theme %%%%%%%%%%%%%%%%%%%%%%
\usetheme{Madrid} % You can use other themes too, but this changes many things. I've found Madrid to be the best for this color scheme

%fg = font color
%bg = background color

% ! WARNING ! : Many colors are linked to multiple attributes, so changing one color can have unexpected changes!

% If you want to tweak the shading of orange and red, tweak the below 2 lines:t
\definecolor{myRed}{RGB}{4,75,4}
\definecolor{myOrange}{RGB}{125, 227, 0}

% Bottom right hand color
\setbeamercolor*{structure}{bg=myRed!20,fg=myRed!90}

\setbeamercolor*{palette primary}{use=structure,fg=white,bg=structure.fg} %?
\setbeamercolor*{palette secondary}{use=structure,fg=myRed,bg=white}
%bottom left of footer & bar between title & top bubbles
\setbeamercolor*{palette tertiary}{use=structure,fg=white,bg=myRed}

\setbeamercolor{frametitle}{bg=myRed!85,fg=white} %title of each slide

\setbeamercolor*{titlelike}{parent=palette primary} %?
%\setbeamercolor{titlelike}{parent=palette primary,fg=structure.fg!50!myRed}

%for miniframe (very top) AND center footer
\setbeamercolor{section in head/foot}{fg=myOrange, bg=white}

%%% Specific Colors %%%
\setbeamercolor{item projected}{bg=myOrange}
\setbeamertemplate{enumerate items}{bg=myOrange}

\setbeamercolor{itemize item}{fg=myOrange}
\setbeamercolor{itemize subitem}{fg=myOrange}

\setbeamercolor{button}{bg=myOrange}

%%% Edits ONLY the TOC slide %%%
\setbeamercolor{section in toc}{fg=black}
\setbeamercolor{subsection in toc}{fg=black}

%%% Block Colors %%%
% Standard block %
\setbeamercolor{block title}{bg=myOrange, fg=white}
\setbeamercolor{block body}{bg=myOrange!20}

% Alerted block % If you want to customize it's color
%\setbeamercolor{block title alerted}{bg=cyan, fg=white}
%\setbeamercolor{block body alerted}{bg=cyan!10}

% Example block % If you want to customize it's color
%\setbeamercolor{block title example}{bg=cyan, fg=white}
%\setbeamercolor{block body example}{bg=cyan!10}

%---------------------------------------------------------
%	SELECT FONT THEME & FONTS
%---------------------------------------------------------
\usefonttheme{default} % Typeset using the default sans serif font
\usepackage{palatino} % Use the Palatino font for serif text
\usepackage[default]{opensans} % Use the Open Sans font for sans serif text
\useinnertheme{circles}

%---------------------------------------------------------
%	SELECT OUTER THEME
%---------------------------------------------------------
% Outer themes change the overall layout of slides, such as: header and footer lines, sidebars and slide titles. Uncomment each theme in turn to see what changes it makes to your presentation.

%\useoutertheme{default}
%
% \useoutertheme{miniframes}

\useoutertheme{infolines}
%\useoutertheme{smoothbars}
%\useoutertheme{sidebar}
%\useoutertheme{split}
%\useoutertheme{shadow}
%\useoutertheme{tree}
%\useoutertheme{smoothtree}

%---------------------------------------------------------
%	PRESENTATION INFORMATION
%---------------------------------------------------------

\title[VIN]{Variational Integrator Networks for Physically Structured Embeddings}
\subtitle{}
\author[]{Author: Igor Alentev}

\institute[]{Robotics Track \\ \smallskip \textit{i.alentev@innopolis.university}}
\date[Fall 2023]
%\date[\today]

\logo{\includesvg[width=2cm]{assets/innou-logo.svg}}

%---------------------------------------------------------
%---------------------------------------------------------
%---------------------------------------------------------
\begin{document}

%---------------------------------------------------------
%	TITLE SLIDE
%---------------------------------------------------------
\section{}
\begin{frame}
    \titlepage % Output the title slide, automatically created using the text entered in the PRESENTATION INFORMATION block above

\end{frame}

%---------------------------------------------------------
%	TABLE OF CONTENTS SLIDE
%---------------------------------------------------------
% References sections and subsections, specified with the standard \section and \subsection commands. If you want to display all sections and subsections on one slide, just use \tableofcontents. If you want to just display each section one at a time (in subsequent slides) use \tableofcontents[pausesections].

\begin{frame}
    \frametitle{Table of Contents} % Slide title, remove this command for no title

    \tableofcontents % Output the table of contents (all sections on one slide)
    %\tableofcontents[pausesections] % Output the table of contents (break sections up across separate slides)
\end{frame}

%---------------------------------------------------------
%	PRESENTATION BODY SLIDES
%---------------------------------------------------------

\section{NDE}

\subsection{Mathematical Formulation}

\begin{frame}
    \frametitle{Neural Differential Equations\footcite{chenNeuralOrdinaryDifferential2019}\footcite{kidgerNeuralDifferentialEquations2022}}
    \framesubtitle{Mathematical Approach}
    What is \textit{Neural Differential Equation} anyway?
    \begin{block}{Neural Differential Equation}
        A \textit{neural differential equation} is a differential equation using a neural
        network to parameterise the vector field. The canonical example is a \textit{neural
            ordinary differential equation}.
        \begin{align}
            y(0) & = y_0 \\ \odv{y}{t}(t) &= f_\theta(t, y(t))
        \end{align}
        Where \(\theta\) is some vector of learnt parameters. Usually, \(f_\theta\) is a feedforward network.
    \end{block}
\end{frame}

\subsection{ML Formulation}

\begin{frame}
    \frametitle{Neural Differential Equations}
    \framesubtitle{Modern Approach}
    \begin{block}{Residual Network (ResNet)}
        \begin{equation}
            y_{j+1} = y_j + f_\theta(j, y_j)
        \end{equation}

        Where \(f_\theta(j, \cdot)\) is \(j\)-th residual block. With \(\theta\) as
        vector of parameters from all layers.
    \end{block}

    If we try the discretization of neural ODE, it might start looking familiar.
    \begin{equation}
        \frac{y(t_{j + 1}) - y(t_j)}{\Delta t} \approx \odv{y}{t}(t) = f_\theta(t_j, y(t_j))
    \end{equation}
    If we absorb the discretization step into the \(f_\theta\), we can derive:
    \begin{equation}
        y(t_{j + 1}) = y(t_j) + f_\theta(t, y_j)
    \end{equation}
\end{frame}
\subsection{Comparison}

\setlength{\columnsep}{0cm}

\begin{frame}
    \frametitle{Neural Differential Equations}
    \framesubtitle{Comparison of RN\footcite{yinMathematicalUnderstandingResNet2019} and NDE}

    \begin{multicols}{2}

        \includegraphics[width=5cm]{assets/nde-rn-comp.png}

        Why to bother with constant discrete number of hidden layers? Continuous-layer
        architecture allows:

        \begin{enumerate}
            \item Precision+Accuracy tuning
            \item Constant memory
            \item Fast backprop
            \item Continuous evaluation
        \end{enumerate}
    \end{multicols}
\end{frame}

\subsection{Conclusion}

\begin{frame}
    \frametitle{Neural Differential Equations}
    \framesubtitle{Coincidences}
    \begin{enumerate}
        \item \textit{Neural ODEs} are the continuous limit of residual networks.
        \item \textbf{GRU} and \textbf{LSTM} updates rules suspiciously similar to
              discretised differential equations.
        \item \textbf{StyleGAN2} is simply discretised \textit{SDE}
        \item Invertible NN coupling layers are reversible DE solvers
    \end{enumerate}
    Many of the DL architectures resemble DEs.
    \begin{block}{}
        Neural Network \(\Leftrightarrow\) Differential Equation
    \end{block}
\end{frame}

\section{VIN}

\subsection{Physically Structured Embeddings}

\begin{frame}
    \frametitle{Variational Integrator Networks\footcite{saemundssonVariationalIntegratorNetworks2020}}
    \framesubtitle{A Bridge}

    VIN is the bridge between the viewpoint of representing deep residual networks
    as discretisation of differential equations and the viewpoint of geometric
    embeddings.

    \begin{block}{Geometric Embeddings}

        \textit{Geometric Embeddings}~\footcite{chamberlainNeuralEmbeddingsGraphs2017}~\footcite{davidsonHypersphericalVariationalAutoEncoders2022}~\footcite{xiongGeometricRelationalEmbeddings2023} is the way of embedding data into it's natural geometry, preserving relational information.

    \end{block}

\end{frame}

\subsection{Variational Mechanics}

\begin{frame}
    \frametitle{Variational Integrator Networks}
    \framesubtitle{Why integrators are variational?\footcite{matthewwestVariationalIntegrators2004}}

    \small
    \begin{block}{ \small Principle of Least Action}
        \begin{equation}
            S(\mathbf{q}) = \int_{0}^{T}L(\mathbf{q}(t), \dot{\mathbf{q}}(t))\mathrm{d}t
        \end{equation}
    \end{block}

    \begin{block}{\small Euler-Lagrange Equations}
        \begin{equation}
            \frac{\mathrm{d}}{\mathrm{d}t}\pdv{L}{\dot{\mathbf{q}}} - \pdv{L}{\mathbf{q}} = 0
        \end{equation}
    \end{block}

    \begin{block}{\small Discrete Euler-Lagrange in Momentum Form}
        \begin{align}
            \mathbf{p}_k       & = -D_1L_d(\mathbf{q}_k, \mathbf{q}_{k+1}, \Delta t) \\
            \mathbf{p}_{k + 1} & = D_2L_d(\mathbf{q}_k, \mathbf{q}_{k+1}, \Delta t)
        \end{align}
    \end{block}

\end{frame}

\subsection{VIN Structure}

\begin{frame}
    \frametitle{Variational Integrator Networks}
    \framesubtitle{Structure}

    \includegraphics[width=15cm]{assets/deep-vin.png}

    Overall structure of the VIN resembles the residual network we seen before. We
    consider it to be a multilayered (discretised) dynamics of the system.
    \((\mathbf{q}, \mathbf{p})\) is the hidden space, which is 2-form of
    position-momentum phase-space and \(f_\theta\) we have seen earlier.

\end{frame}

\subsection{Benefits}

\begin{frame}
    \frametitle{Variational Integrator Network}
    \framesubtitle{Benefits}

    \begin{enumerate}
        \item Physical properties conservation automatically enforced by underlying structure

              \item{Interpretability
                          is increased.

                          Embedding evolves in the phase-space, which has notions of kinetic and
                          potential energy. }
        \item Complex phenomena modelling is allowed due to the black-box nature of
              potential.
    \end{enumerate}

\end{frame}

\subsection{VIN-VAE}

\begin{frame}
    \frametitle{Variational Integrator Networks}
    \framesubtitle{VIN-VAE}
    Case study: \textit{Ideal pendulum}. Noisy observations. Image data.

    \includegraphics[width=15cm]{assets/vin-vae.png}

    What if we will consider VIN in terms of VAE\@?

\end{frame}

\subsection{Comparison}

\begin{frame}
    \frametitle{Variational Integrator Networks}
    \framesubtitle{Case Study}

    \begin{multicols}{2}
        \includegraphics[width=7cm]{assets/rnn-vin.png}

        Let's infer and evolve the embeddings of ResRNN and VIN\@. The main question is
        the preservation of evolution patterns.
    \end{multicols}

\end{frame}

\section{Conclusion}

\begin{frame}
    \frametitle{Conclusion}
    \centering
    \LARGE{Conclusion \& Questions}

\end{frame}

% \section{Motivation} % Note all sections and subsections are automatically placed in your table of contents

% %------------------------------------------------
% \begin{frame}
%     \frametitle{Title}
%     \begin{center}
%         \textbf{To use this template, you can copy and just edit/add slides!}\newline
%     \end{center}

%     This is because all of the color customization occurs in the "Customize Themes"
%     section in lines 12-51 of the code\newline

%     The remainder of these slides serve as an example to show all the features you
%     can use: bullets, buttons, sections, etc.

%     \begin{center}
%         \emph{This was a labor of love, I hope you like it!}
%     \end{center}
% \end{frame}

% %------------------------------------------------
% \begin{frame}
%     \frametitle{Another Title}
%     \framesubtitle{and a subtitle!}

%     Look at the code of this slide to see how columns made this formatting look
%     nice.

%     \begin{columns}[t] % The "c" option specifies centered vertical alignment while the "t" option is used for top vertical alignment
%         \begin{column}{0.5\textwidth} % Right column width
%             % To add an image %
%             \begin{figure}[h!]
%                 \centering
%                 %\caption{Hawkins et al, 2015}
%                 \includegraphics[angle=0, width=4.5cm]{Hokie2.png}
%                 %\label{Figure 1}
%             \end{figure}
%         \end{column}
%         \begin{column}{0.5\textwidth} % Left column width
%             \begin{figure}[h!]
%                 \centering
%                 %\caption{Hawkins et al, 2015}
%                 \includegraphics[angle=0, width=4.5cm]{Hokie2.png}
%                 %\label{Figure 1}
%             \end{figure}
%         \end{column}
%     \end{columns}
% \end{frame}

% %------------------------------------------------
% \begin{frame}
%     \frametitle{Yet another title}
%     You can use bullets too:\newline
%     \begin{itemize}
%         \item Like this one\newline
%         \item \& this one
%     \end{itemize}
% \end{frame}

% %------------------------------------------------
% \begin{frame}
%     \label{Test} %For the link button for the Appendix slide
%     \frametitle{A title}

%     \begin{itemize}
%         \item You can also nest sub-bullets
%               \begin{itemize}
%                   \item Sub-bullet 1
%                   \item Sub-bullet 2
%                   \item Sub-bullet 3
%                   \item Sub-bullet 4 \newline
%               \end{itemize}
%     \end{itemize}

%     You can add citations \footcite{Tjøstheim} too\newline

%     \textbf{Below is a button that links to a slide in the appendix}

%     \begin{center}
%         \hyperlink{Figure}{\beamergotobutton{Go to graphs}}
%     \end{center}
% \end{frame}

% %------------------------------------------------
% \section{Theory}

% %------------------------------------------------
% \begin{frame}
%     \label{Test Stat}
%     \frametitle{The Test Statistic}

%     Here is a made up equation: $$ \hat{A} = \bar{m}-\hat{m}_S$$ \newline

%     Notice how these buttons are centered and evenly spread out:\newline

%     \begin{columns}[t] % The "c" option specifies centered vertical alignment while the "t" option is used for top vertical alignment
%         \begin{column}{0.25\textwidth} % Right column width
%             \hyperlink{Terms}{\beamergotobutton{Go to Terms}}
%         \end{column}
%         \begin{column}{0.25\textwidth} % Left column width
%             \hyperlink{Definitions}{\beamergotobutton{Go to Definitions}}
%         \end{column}
%         \begin{column}{0.25\textwidth} % Left column width
%             \hyperlink{Theorems}{\beamergotobutton{Go to Theorems}}
%         \end{column}
%     \end{columns}

% \end{frame}

% %------------------------------------------------
% \begin{frame}
%     \frametitle{No way, another title!}
%     \begin{enumerate}
%         \item Instead of bullets, you can index by number too\newline
%         \item like this
%     \end{enumerate}
% \end{frame}

% %------------------------------------------------
% \section{Testing}

% %------------------------------------------------
% \begin{frame}
%     \frametitle{Second to last title}
%     \begin{block}{Block Title}
%         Block 1
%     \end{block}

%     \begin{exampleblock}{Example Block Title}
%         Block 2
%     \end{exampleblock}

%     \begin{alertblock}{Alert Block Title}
%         Block 3
%     \end{alertblock}

%     \begin{block}{} % Block without title
%         Block without a title
%     \end{block}
% \end{frame}

% %------------------------------------------------
% \section{Conclusion}
% %------------------------------------------------
% \begin{frame}
%     \frametitle{Last title}

%     Last bit of text

% \end{frame}

%---------------------------------------------------------
%	CLOSING SLIDE
%---------------------------------------------------------

% To remove miniframe from top
% \appendix
% \setbeamertemplate{headline}{}
% \addtobeamertemplate{frametitle}{\vspace*{-\headheight}}{}

% \begin{frame}[noframenumbering] %So the end and appendix slides don't contribute to the page count
%     %[plain] % The optional argument 'plain' hides the headline and footline
%     %\frametitle{Questions?}

%     \begin{center}
%         {\LARGE Questions?}
%     \end{center}

% \end{frame}

%---------------------------------------------------------
%	REFERENCES
%---------------------------------------------------------

\begin{frame}[noframenumbering, allowframebreaks]
    \frametitle{References}

    \printbibliography
\end{frame}

%------------------------------------------------
% \begin{frame}[noframenumbering]
%     \label{Figure}
%     \frametitle{Appendix - A figure}
%     \hyperlink{Test}{\beamerreturnbutton{Return to presentation}}

%     \begin{figure}[h!]
%         \centering
%         %\caption{}
%         \includegraphics[angle=0, width=5cm]{Newey et al Graph.png}
%         %\label{fig}
%     \end{figure}
% \end{frame}

% %------------------------------------------------
% \begin{frame}[noframenumbering]
%     \label{Terms}
%     \frametitle{Appendix - Terms}

%     \begin{columns}[t] % The "c" option specifies centered vertical alignment while the "t" option is used for top vertical alignment
%         \begin{column}{0.5\textwidth} % Right column width
%             Some Estimators:
%             \begin{itemize}
%                 \item Drift: $\hat{\delta}$
%                 \item Boundary: $\hat{b}(t)$
%             \end{itemize}
%         \end{column}
%         \begin{column}{0.5\textwidth} % Left column width
%             Some Variables:
%             \begin{itemize}
%                 \item $\hat{V}$
%                 \item $\hat{m}_S$
%                 \item $\bar{m}$
%                 \item $m_J(\tau)$\newline\newline
%             \end{itemize}
%         \end{column}
%     \end{columns}
%     \hyperlink{Test Stat}{\beamerreturnbutton{Return to presentation}}
% \end{frame}

% %------------------------------------------------
% \begin{frame}[noframenumbering]
%     \label{Definitions}
%     \frametitle{Appendix - Definitions}
%     \begin{enumerate}
%         \item A definition \newline
%     \end{enumerate}

%     \hyperlink{Test Stat}{\beamerreturnbutton{Return to presentation}}
% \end{frame}

% %------------------------------------------------
% \begin{frame}[noframenumbering]
%     \label{Theorems}
%     \frametitle{Appendix - Theorems}
%     \begin{enumerate}
%         \item A theorem\newline
%     \end{enumerate}

%     \hyperlink{Test Stat}{\beamerreturnbutton{Return to presentation}}
% \end{frame}

\end{document}